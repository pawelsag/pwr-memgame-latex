\documentclass[polish,polish,a4paper]{article}
\usepackage[T1]{fontenc}
\usepackage[utf8]{inputenc}
\usepackage{babel}
\usepackage{pslatex}
\usepackage{graphicx}
\usepackage{tikz}
\usepackage{pgfplots}
\usepackage{anysize}
\usepackage{pgfgantt}
\usepackage{tabularx}
\usepackage{float}
\usepackage{latexsym,amsmath}
\marginsize{2.5cm}{2.5cm}{3cm}{3cm}
\usepackage{hyperref}
\newcommand{\name}[1]{\sffamily\bfseries\scriptsize #1}
\newcommand{\frontpage}[8]{
%% #1 - nazwa kursu
%% #2 - kierunek 
%% #3 - termin 
%% #4 - temat 
%% #5 - problem
%% #6 - skład grupy
%% #7 - nr grupy
%% #8 - data

%\hfill\includegraphics[width=4cm]{PWr.png}
\vspace{2cm}

\begin{tabular}{|p{0.72\textwidth}|p{0.28\textwidth}|}
\hline
\multicolumn{2}{|c|}{}\\
\multicolumn{2}{|c|}{{\LARGE #1}}\\
\multicolumn{2}{|c|}{}\\
\hline
\name{Kierunek} & \name{Termin}\\
\multicolumn{1}{|c|}{\textit{#2}} & \multicolumn{1}{|c|}{\textit{#3}} \\
\hline
\name{Temat} & \name{data}\\
\multicolumn{1}{|c|}{\textit{#4}} & \multicolumn{1}{|c|}{\textit{#5}} \\
\hline
\name{Grupa oddająca projekt} & \name{Nr indeksu}\\
\multicolumn{1}{|c|}{\textit{#6}} & \multicolumn{1}{|c|}{\textit{#7}} \\
\hline
\end{tabular}
}

\usepackage{listings}
\usepackage{xcolor} % for setting colors

% set the default code style
\lstset{
    frame=tb, % draw a frame at the top and bottom of the code block
    tabsize=4, % tab space width
    showstringspaces=false, % don't mark spaces in strings
    numbers=left, % display line numbers on the left
    commentstyle=\color{green}, % comment color
    keywordstyle=\color{blue}, % keyword color
    stringstyle=\color{red} % string color
}

\title{Sprawozdanie SPD}
\pgfplotsset{compat=1.16}
\begin{document}
\includegraphics[width=6cm]{PWr.png}
% #1 - nazwa kursu #2 - kierunek  #3 - termin #4 - temat #5 - data #6 - skład grupy #7 - nr grupy
\frontpage{Układy cyfrowe i systemy wbudowane}{Informatyka}{Poniedziałek 11:05}{Gra pamięciowa}{\today}{Paweł Sagan, Patryk Wlazłyń}{241250, 241249}
\pagestyle{}
\newpage
\tableofcontents
\newpage
\section{Wstęp}
Celem projektu było zaprojektowanie oraz wykonanie programu dedykowanego pod układ FPGA, a następnie jego przetestowanie na układzie Spartan-3E.
Wybrany przez nas temat to gra pamięciowa, w której gracz musi zapamiętać umiejscowione liczby na ekranie.
Gdy gracz naciśnie pierwszą z liczb, reszta zamienia się w nierozróżnialne pola (wypełnione prostokąty).
Celem gracza jest zapamiętanie układu liczb, oraz kliknięcie pól w kolejności jaką reprezentowały wcześniej wyświetlone liczby.

\section{Założenia projektowe}
Poniżej lista zawierająca przedstawione założenia projektowe:
\begin{itemize}
  \item Gra dedykowana jest dla jednego gracza
  \item Podstawowy interfejsem gracza jest mysz podłączona za pomocą złącza ps2
  \item Aktualny stan gry oraz komunikaty wyświetlane są na ekranie którego rozdzielczość renderowanego obrazu wynosi 48 linii po 20 znaków
  \item Początkowe umiejscowienie liczb na ekranie, losowane jest przy użyciu algorytmu pseudolosowego LCG.
  \item Pierwsza poprawną liczbą jest wartość najmniejsza spośród wyświetlonych.
  \item Ukrycie liczb następuje w momencie kliknięcia na pierwszą z nich.
  \item Odkrycie liczby większej o dwa od poprzednio odkrytej liczby, powoduje zakończenie gry oraz wyświetleniu komunikatu o przegranej.
  \item Użytkownik wygrywa gdy wszystkie liczby zostaną poprawnie odkryte. Wówczas zostaje wyświetlony stosowny komunikat.
\end{itemize}

\section{Opis projektu}
Program składa się z modułów wykonywanych w języku VHDL, które następnie połączone zostały za pomocą głównego schematu projektu.
Oprócz modułów wykonanych na potrzebę tego programu, użyte zostały komponenty dostarczone na stronie kursu, konkretnie PS2\_MOUSE oraz VGAtxt48x20.
Ze względu na ograniczony dostęp do sprzętu, dla każdego komponentu utworzony został test, który sprawdza jego poprawną implementacje.
Dodatkowo aby zminimalizować ilość błędów, komponenty zostały pogrupowane w zbiory zawierające najmniejsza możliwą ilość modułów stanowiącą logiczną całość.
Dzięki takiemu podejściu analiza błędów syntezy jest mniej złożona.
Projekt dedykowany jest pod zestaw Digilent S3E-Starter, który to posiada wyprowadzone złącze PS2 oraz konektor VGA.
\newline
Głównym komponentem programu jest moduł logic który spaja ze sobą wszystkie stworzone moduły.
W poniższej sekcji znajduje się dokładny opis zaprojektowanych modułów oraz relacji pomiędzy nimi. 
\newpage
\subsection{Wykorzystane moduły}
\subsubsection{Stworzone moduły}
\begin{itemize}
 \item
  Licznik - Prosty 16-bitowy moduł zliczający cykle zegara oraz podający informacje o aktualnym stanie licznika na magistrale.
  Użyty jako 'seed' dla modułu implementującego generowanie liczb pseudolosowych.

 \item
  PS2\_Controler - Moduł służy do wyliczania aktualnej pozycji myszki.
  Wyliczone dane ograniczone są przez minimalną oraz maksymalną rozdzielczość ekranu.
  Oznacza to, że moduł nigdy nie zwróci ujemnej pozycji oraz pozycji większej niż założona.
  Układ propaguje również informacje o kliknięciu przez użytkownika lewego przycisku myszy.
  Informacja widnieje na linii tak długo, aż inny moduł nie poprosi o jej wyczyszczenie.
  Wystawione dane pozycji myszki w momencie kliknięcia, pozostają niezmienione, aż do momentu ustawienia linii na stan niski. 

 \item
  vga\_driver - Prosty układ potrafiący przekazywać dane od logiki programu na wejście sterownika VGAtxt48x20.
  Układ posiada 4 tryby.
  Pierwszy tryb umożliwia renderowanie pamięci bez ukrywania jej zawartości.
  Oznacza to, że wszystkie wygenerowane liczby renderowane są jako ich odpowiadająca znakowa reprezentacja.
  Kolejny tryb umożliwia renderowanie ukryte.
  Oznacza to, że wszystkie liczby w pamięci wyświetlane są jako białe kafelki.
  Dwa ostanie tryby umożliwiają wyświetlenie informacji o stanie gry użytkownika.
  Pierwszy z nich renderuje informacje o przegranej użytkownika, drugi zaś o wygranej.
  Moduł przyjmuje informacje o pozycji myszki, dzięki czemu jest w stanie wygenerować symbol myszy w odpowiednim miejscu.

 \item
  lcg - Moduł odpowiedzialny za generowanie liczb pseudolosowych.
  Wewnętrznie używa implementacji algorytmu LCG (Linear congruential generator).
  Parametry algorytmu zostały dobrane eksperymentalnie, bazując na rozwiązaniach w bibliotece glibc (standardowa biblioteka C, GNU).
  Dobrane parametry zostały zasymulowane w języku Python, aby potwierdzić równomierny rozkład losowanych liczb.

 \item
  object\_generator - Moduł odpowiedzialny za wygenerowanie kolejnych obiektów wyświetlanych na ekranie w losowych miejscach.
  Wewnętrzna implementacja korzysta ze zmodyfikowanego algorytmu "Fisher–Yates shuffle".
  Aby potwierdzić równomierny rozkład podczas operacji mieszania, zmodyfikowany algorytm został w pierwszej kolejności zaimplementowany w języku Python.
  Modyfikacja polega na poszerzeniu zakresu losowania elementów, które zamieniane są miejscami, na stałą z góry zadaną wartość.
  Pozwala to na całkowite pozbycie się modułu dzielącego.
  Dzielenie przez stałą zostaje wyoptymalizowane jako mnożenie i dodawanie.
  object\_generator operuje na pamięci ram w celu umiejscowienia liczb w losowo wybranych miejscach.
  Parametr wielkości miejsc pustych nie może być zmieniony dynamicznie (w trakcie działania układu), natomiast jest łatwo modyfikowalny w kodzie.
  Wymaga to jednak ponownej syntezy.
  Parametr ilości generowanych obiektów zajmujących nie puste pola jest natomiast całkowicie konfigurowalny w czasie działania układu.

 \item
   ram16 - prosty moduł implementujący pamięć RAM z 16-bitową szyną adresową dla 8-bitowych komórek danych.

 \item
   memmux - prosty moduł implementujący multiplexer dostępu do modułu ram16.

 \item
  logic - główny moduł spajający i zarządzający.
  Implementuje główną logikę gry, oraz zarządzanie jej stanem.
  Steruje restartowaniem oraz inicjalizacją gry, przełącza moduł vga\_driver w odpowiedni tryb wyświetlania, obsługuje zdarzenia generowane przez moduł 
  PS2\_Controler, oraz zarządznie dostępem do współdzielonej magistrali do pamięci RAM za pomocą multiplexera memmux.


\end{itemize}

\subsubsection{Zewnętrzne moduły}
\begin{itemize}
 \item
  Moduł PS2\_Mouse - Moduł ten przekazuje dane dostarczone przez aktualnie podłączoną myszkę.
  Kiedy dane są aktywne układ ustawia jedno cyklowy stan wysoki na linii.
  Oprócz tego dostarcza informacje o przesunięciu myszki względem ostatniej pozycji w osi X oraz Y.
  Moduł przekazuje informacje na temat zwrotu myszki w poszczególnych kierunkach oraz informacje na temat wciśniętych przycisków.
  Oś X, Y oraz status są 8 bitowymi magistralami.

 \item
  Moduł VGAtxt48x20 - moduł wyświetla przekazane dane na ekranie monitora.
  Dane podawane są na 8 bitową magistrale.
  Moduł wyświetla znaki zakodowane według standardowej tablic ASCII, wyjątkiem są wartości z przedziału 0x0 - 0x1F.
  Informacje na temat wspominanego zakresu można znaleźć w dokumentacji modułu.
  Po wyświetleniu zadanego znaku, moduł automatycznie zwiększa pozycję na której wystąpi wyświetlenie kolejnego znaku.
  Możliwy jest również powrót kursora do początku linii lub początku ekranu.

\end{itemize}
\subsection{Wynik symulacji wybranych modułów }
\subsubsection{Object generator}
\includegraphics[width=16cm]{object_generator_tb.png}
\break
Object generator odpowiedzialny jest za umieszczenie kolejnych symboli w losowych miejscach na ekranie.
Aby to osiągnąć wykorzystuje bezpośrednio moduł lcg do generowania liczb losowych, oraz moduł ram16
do przechowywania informacji na temat generowanych obiektów.
Ostatecznym wynikiem jest tablica N obiektów (gdzie N zadawane jest jako parametr dynamiczny) ułożonych
w pamięci w sposób losowy na M pozycjach (gdzie M jest wielkością wyświetlanego obrazu, wyrażoną w obiektach).
Było to głównym powodem wyboru tego modułu, by na jednym przebiegu czasowym pokazać działanie wszystkich wspomnianych komponentów.
Po 35ns na wejście i\_object\_count modułu object\_generator przekazywana jest ilość obiektów do wygenerowania.
W tym przypadku są to 4 obiekty.
Układ dodatkowo jest resetowany.
Po wyłączeniu sygnału reset wejście i\_random zaczyna podawać pseudolosowe sekwencję, które przekazywane
są bezpośrednio z modułu lcg.

Algorytm losowania miejsc dla obiektów działa na podstawie algorytmu "Fisher–Yates shuffle".
Na początku, od 65ns do 105ns, w stanie "0" następuje zapisanie obiektów w kolejnych komórkach pamięci.
W tym przypadku są to obiekty pod adresami od 0 do 3. Wpisane wartości to odpowiednio
kody liter ASCII kolejnych cyfr.

Następnie moduł, w stanie "10" rozpoczyna iterowanie po całej, tak wygenerowanej tablicy, oraz
zamianie zawartości z losowymi komórkami pamięci.
Procedura swap(*addr1, *addr2) osiągana jest poprzez dwa odczyty oraz dwa zapisy pod zamienione adresy.
Taką pojedynczą wymianę, można zaobserwować na osi w od czasu 115ns do 155ns.
Najpierw odczytywana (o\_mem\_we = 0) jest wartość spod adresu 0x0001.
W następnym cyklu odczytana zostaje wartość 0x49.
Następnie następuje odczyt z losowego adresu, w tym przypadku 0x0063.
Zaraz po nim widać dwa zapisy do pamięci (o\_mem\_we = 1) pod adresy 0x0063 oraz 0x0001.
Jedyna różnica jest teraz taka, że wcześniej odczytane wartości są teraz wpisane w odwrotnej kolejności
tj. *0x0063 = 0x49, *0x0001 = 0x00.
Taka procedura powtarzana jest N razy.
Po niej ustawiane jest wyjście o\_ready = 1, które oznacza, że moduł zakończył generowanie obiektów
i logika może przejść do kolejnej fazy. Wyjście ze stanu o\_ready = 1 możliwe jest poprzez ponowne ustawienie
sygnału i\_reset = 1.

W powyższym przykładzie został pominięty wykres zerowania całej pamięci, które ma miejsce zaraz przed
rozpoczęciem generowania nowych elementów, dla zachowania większej czytelności.
Zerowanie jest wymagane, aby można było w dowolnym momencie zresetować cały stan gry i rozpocząć na nowo,
np. w przypadku przegranej.
\subsubsection{PS2 Controller}
\includegraphics[width=16cm]{ps2_tb.png}
\break
Moduł PS2 Controller wykorzystuje komponent PS2 udostępniony w ramach kursu układów 
Cyfrowych i systemów wbudowanych do odczytywania przesunięcia myszki
oraz status przycisków. 
Moduł wylicza pozycję kursora dbając o to aby nie przekroczył on założonych granic. 
Pozycja kursora nie może znaleźć się poza ekranem, stąd jego najmniejsza wartość jest 0.
W zależności od wartości magistrali 'Status' która niesie informacje o kierunku ruchu myszki,
wartości w osi X są odejmowane gdy Status(5) = 1 oraz dodawane gdy Status(5) = 0.
Analogicznie odbywa się to dla osi Y.
Przygotowany testbench w nieskończonej pętli odejmuje oraz dodaje na zmianę zmieniające się wartości z tablicy.
Takie zachowanie wymuszone jest negującą się wartością wszystkich bitów na magistrali 'Status', które niesie informacje o kierunku
ruchu myszki.
Są to kolejno [0x1, 0x2, 0x3, 0x4]. Testbench pokazuje, że wynik odjęcia pierwszej wartości od zera ulega saturacji
nie przekraczając z góry założonych ograniczeń.
Następnie dodajemy 2, odejmujemy 3, dodajemy 4, po czym znów odejmujemy 1.
Pozycja myszki przechowywana jest na dwóch magistralach "x\_cur\_pos" oraz "y\_cur\_pos".

\subsection{Osiągnięte cele}
Wszystkie moduły do zrealizowania kompletnej logiki gry zostały utworzone oraz przetestowane przy użyciu symulatora.
Proces generowania bit-streamu oraz syntezy przebiegł pomyślnie.
Ze względu na ograniczony dostęp do sprzętu niemożliwe jest przetestowanie układu na sprzęcie.

\section{Instrukcja obsługi}
Po załadowaniu programu na urządzenie, gracz rozpoczyna nową grę przez kliknięcie w dowolnym miejscu na ekranie.
Następnie stan nowej gry zostaje zainicjalizowany w pamięci, oraz zostają wylosowane miejsca na kolejne liczby.
Liczby zostają ukazane na ekranie, a zadaniem gracza jest kliknięci ich w kolejności reprezentowanej przez wartości liczb.
Kliknięcie chociaż jednej z nich w złej kolejności spowoduje przegraną (aby rozpocząć na nowo, należy kliknąć w dowolne miejsce na ekranie).
Aby wygrać wszystkie obiekty muszą zostać odkryte.

\section{Podsumowanie}
Projektowanie aplikacji oraz tworzenie ich w języku VHDL wymaga podejścia innego niż w przypadku programowania w językach wysokiego poziomu, które
wykonywane są przez procesory ogólnego przeznaczenia.
Należy pamiętać o zależnościach czasowych oraz o tym, że łączenie instrukcji oferowanych przez język VHDL nie odzwierciedla zachowania znanych z innych wysokopoziomowych języków programowania, jak np. to, że rezultaty niektórych przypisań widoczne są dopiero z następnym cyklem zegara.
\newline
Kod projektu zamieszczony jest w serwisie github dostepny po kliknięciu na wyróżniony tekst:
\textcolor{blue}{\href{https://github.com/3919/pwr-memgame}{Pwr-memgame}}
\end{document}
